\subsection{Real-time Social Network Analysis}

\mnote{Case study for queries with multiple results}

% 
% \section{Twitter Analisys Tools}
% 
% Many tools have been developed to extract information from Twitter statuses. Th
% 
% 
% \emph{Social Mention}~\cite{socialmention} is a tool that allow the tracking of certain keywords across
% over 80 social networks. It monitors the status update streams produced by users of the major social
% networks and it filters only those containing a certain keyword. It provides a list of all statuses
% together with a number of statistics. It is possible for instance to monitor how many people mention the
% keyword, what sentiment are expressed in the statuses, what other keywords appear the most together with
% the status. One of the aims of this tools is to help marketers to assess the perception of a brand or to
% measure the impact of media campaign. 
% \emph{Strength} is the likelihood that a certain brand is being discussed on social media, given by
% phrase mentions within the last 24 hours divided by total possible mentions. \emph{Sentiment} is the
% ratio of mentions that are generally positive to those that are generally negative. \emph{Passion} is a
% measure of the likelihood that individuals talking about a brand will do so repeatedly. \emph{Reach} is 
% a measure of the range of influence. It is the number of unique authors referencing a brand divided by
% the total number of mentions. 
% 
% 
% 
% 			\bul{Overview of trends in social networks}
% 			% \incite{social-networks}
% 
% 			\incite{storm} 
% 
% 			\bul{Twitter Analysis Tools}
% 			%\incite{socialmention}
% 			\incite{twitrratr}
% 			\incite{twitter-sentiment}
% 			\incite{ubervu}
% 			\incite{twendz}
% 			\incite{tweetfeel}
% 
% 			\bul{FourSquare Articles}
% 			\incite{foursquare-wsj}
% 			\incite{foursquare-rude}	
% 
% 			\bul{Papers on Twitter sentiment analysis}
% 			\incite{twitter-stocks}
% 			\incite{twitter-sentiment-1}
% 			\incite{twitter-sentiment-2}
% 			\incite{twitter-sentiment-3}
% 			\incite{twitter-sentiment-4}
% 			\incite{twitter-sentiment-5}
% 
% 			\bul{Papers on Social Casacades}
% 			\incite{buzztraq}
% 			\incite{socialcascade-flickr}
% 			\incite{socialcascade-salvo}
% 
% 
% % social cascades
% \section{Social Cascades}
% One interesting application of this social media real-time analysis is the possibility of predicting
% social cascades. A \emph{social cascade} is the phenomenon generated by the repetitive sharing of a
% certain content over Online Social Networks (OSNs). One person discovers an interesting piece of
% information and shares a link to it with a few friends, who share it themselves and so on. When a content
% is considered interesting by a community it starts being shared over and over again, possibly reaching a
% large number of hits in a short period of time~\cite{socialcascade-flickr}. Because of the
% characteristics of social networks this process mimics the spread of an epidemics~\cite{buzztraq}.
% Initial studies about this phenomenon date back to the 1950s~\cite{diffusion-innovation}, with the theory of Diffusion
% of Innovation. It is only now though, with the wealth of information shared through OSNs that is possible
% to study social cascading in much more detail.
% 
% More formally we can say that a user \emph{Bob} was reached by a social cascade when is reached by a
% certain content \emph{c} when:
% \begin{enumerate}
% 	\item User \emph{Alice} already posted content \emph{c} before user \emph{Bob and}
% 	\item There is a social connection between user \emph{Alice} and user {Bob}
% \end{enumerate}
% 
% 
% 
% \paragraph{Social Cascading in Twitter}
% The diffusion of links through Twitter can be used to better understand how a social cascading tree looks
% like. Twitter is a micro-blogging website where users can share a short message of maximum 140 characters
% with other users called followers. In this particular social network it is also possible to further
% characterise cascades into to main groups: \textbf{L-cascades} and
% \textbf{RT-cascades}~\cite{outweeting}.
% 
% \textbf{L-cascades} occuour when a certain content is shared by direct followers. More formally we can
% say that an L-cascade is the graph of all users who tweeted about a certain content \emph{c}.A cascade
% link is formed when 1) User \emph{Alice} and user \emph{Bob} shared a content \emph{c}, 2) User
% \emph{Alice} posted \emph{c} before user \emph{Bob}, 3) User \emph{Bob} is a follower of user
% \emph{Alice}.
% 
% \textbf{RT-cascades} instead do not only take into account only direct connections, but also the
% possibility of crediting a certain content to a user even without being a direct follower. If user
% \emph{Bob} wants to give credit to user \emph{Alice} for a certain content, he prepends his new tweet
% with the conventional \emph{RT @Alice} followed by the original tweet. In this way \emph{Bob} gives a
% direct credit to \emph{Alice} for the content of the tweet even without being a direct follower. This
% phenomenon became known as retwitting~\cite{}. More formally we can say that an RT-cascade \emph{R(c)} is
% the graph all the users who have retweeted content \emph{c} or have been credit for it. A cascade link is
% formed when 1) User \emph{Bob} tweeted about content \emph{c}, 2) User \emph{Alice} tweeted about content
% \emph{c} before user \emph{Bob}, 3) User \emph{Bob} credited user \emph{Alice} as the original source of
% the content. 
% 
% \paragraph{Impact of Social Cascades on CDNs}
% % CDN
% It is difficult to predict where and when a certain content will become popular, 
% Many content providers rely on Content Delivery Networks (CDNs) to distribute their content across
% several geographically distributed locations in order to enhance the fruibility of the content by their
% users. The choice of \emph{what} content to replicate, \emph{where} and for \emph{how long} is crucial to
% the reduction of the costs associate with the use of a CDN. Being able to predict the popularity trend of
% a certain content is needed for the correct provisioning of resources. This is especially important since
% it has been found~\cite{bz1} that the top 1top 10\% of the videos in a video-on-demand system account for
% approximately 60\% of accesses, while the rest of the videos (the 90\% in the tail) account for
% 40\%~\cite{buzztraq}.
% \\
% \todo{buzztraq}
% 
% Another attempt to characterise the spread of social cascades focuses on improving the caching of
% contents exploiting the geographic information contained in the cascades~\cite{socialcascades-salvo}. 
% This approach leverage the observtion that a social cascade tends to propagate in a geographically
% limited area, since social connections often reflect real life relationships. It easy to see how a video
% spoken in a particualar national language would tend to propagate within the national borders of the
% country in which that language is native. This study analises a corpus of 334 millions messages shared on
% Twitter, extracting about 3 millions single messages with a video links. It was found that about 40\% of 
% steps in social cascades involve users that are, on average, less than 1,000 km away from each others.
% 
% 
