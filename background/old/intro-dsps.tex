%good replication
The natural step in the evolution of DSPSs is distribution. Centralized systems are inherently subject to scalability
issues, being able to distributed the processing over a number of nodes is the logical approach to overcome this issue.
Distribution also allow for better dependability as operators can be replicated at different locations maximising
availability. Fault-tolerance can also be better achieved by employing distribution, as different replicas of operators
can be deployed, achieving resilience to failure. The overall performance of the system is also increased by this
approach, as the computation can be partitioned and run in parallel at different computing nodes. Distribution can be
realised at different scales. Processing nodes can be located within the same datacenter, taking advantage of the 
high-bandwidth and low-latency typical of these environment, or spanned over wide-area networks to push scalability even
further.

%issues
Distribution presents several advantages, but it does come at a cost. Many issues arise when the system is decentralized: 
the operational complexity the system increases and resources need to be allocated efficiently. In a DSPS distribution is 
achieved by partitioning the query plan and running clusters of operators at different sites. Placing these operators
correctly at deployment time and moving them at run time to rebalance the system is not an easy task. Running replicas,
also, can greatly help to increase the dependability of the system, but their correct management also represent a key
issue. Finally, being able to fully exploit the system resources is also major challenge. 

\subsection{Operator Placement and Load Balancing}

%op placement is difficult
The optimal distribution of streaming operators over a number of network machines is key to maximise the performance of a
decentralize stream processing system. Once the query plan has been divided into clusters, these needs to be mapped to
some available physical computing sites. A placement algorithm assigns operators to processing nodes while satisfying a
set of constraints and attempts to optimize some objective function. Finding the optimal assignment among the total
possible assignments is an NP-complete problem and thus computationally intractable~\cite{npc-placement}. Despite this,
many strtegies has been devised for efficient operator placement and depending on the assumptions and requirements of the
system, different approaches can be more suitable than others. A taxonomy of various placement strategies for operators
can be found in~\cite{placement_strategies}.

%load-balancing
Closely tied to operator placement is the problem of load balancing. While the former is more concerned with the
placement at deployment time, load balancing has to deal with the moving of operator at run time. During the
execution of the query the amount of data carried by the stream can greately vary and this may result in computational
overload at some nodes. To recover from this overload condition, the system can decide to migrate some operators to
better location, attempting to balance the load among the available resources. Many operator placement algorithms
recognize the need of placement reconfiguration at run time and make load balancing a key component of their strategy.

%centralized
When the processing nodes of a DSPS are located within the same datacenter, the topology management can be assigned to a
centralised placement controller. In these environments the controller can easily be aware of if the current state for
the entire network, including workload information and resource availability. Having access to this information allows
it to reason about placement choices, with results that are potentially optimal for small deployments. When the number
of available resources becomes very large though, these strategy are not indicated because of their low degree of
scalability. Abadi et al.~\cite{borealis-design} propose an example of such a solution, which assumes a fairly constant
workload and an heavy cost for operator migration. Xing et al.~\cite{borealis-xing_placement} instead, consider the workload to
be higly unpredictable, thus requiring some load balancing at runtime. They also acknolodge the importance of initial
deployment, and a high migration cost for operators, thus developing an algorithm resilient to load
variations~\cite{borealis-load}. All these algorithms have been implemented within the Borealis DSPS. 

%decentralized
When processing nodes are instead distributed over wide-are networks, a central coordinator becomes ineffective and a 
decentralized approach to the problem have to be employed. These algorithms make decisions based on local workload
and resource information. Pietzuch et al.~\cite{sbon}, for instance, employ a distributed algorithm based on multiple spring
relaxation, trying to globally minimize a metric called network usage, based on both bandwith and latency. Another
approach has been taken by Amini et al.~\cite{extreme-scale-sps}, in this case the algorithm maximises weighted throughput of
processing elements, while ensuring stable operation in the face of highly bursty workloads. Zhou et al.~\cite{placement-zhou} 
propose an algorithm in which the initial deployment is determined by minimising the communication cost among
processing nodes, and then adapts to the changes in the data and workload within the system. Ying et al.~\cite{placement-ying}
propose to also account for the state of operators when performing migration decisions. When the resources are
administered by multiple authorities, the degree of trust and collaboration among them must be taken into account. 
The algorithm from Balazinska et al.~\cite{medusa-load} achieve this by the mean of of prenegotiated load-exchange contracts
between the nodes. 

\subsection{Operator Replication}
% benefits of replication
Operator replication enables DSPSs to reduce the impact of system and network failures. Efficient replica allocation and
management is of fundamental importance to maximise the dependability and availability of the system. Replication helps
the system to be resilient to failures, as the computation of a crashed node is mirrored by one or more replicas, and can also
overcome network partitions by exploiting geographic diversity. In general the system tries to allocate as many
replicas as possible, given the amount of available resources.  As DSPSs move towards an Internet-scale, resource
overprovisioning becomes more and more unfeasible, stressing even further the necessity of intelligent replica management.
Also, for these extremely large systems, failure is much more likely to occur, becoming the usual operational mode
rather than the exception. Correctly managing replicas, though, is a real challenge.

%borealis
Early works on replication in a DSPS have been focusing on masking software failure~\cite{borealis_ha_algos}, by running multiple copies
of the same operators at different nodes. Others~\cite{ha_ft_dataflows} strictly favoured consistency over availability, by requiring at least
one fully connected copy of the query tree to be correctly working in order for the computation to proceed. \\
In Borealis, the approach has been to provide a configurable trade-off between availability and
consistency~\cite{borealis-fault_tolerance}.
This is done by letting the user specify a time threshold, within this time window all tuples are processed regardless
of whether failures occur or not on input streams. This increase availability, since the operator continues to process
even in presence of failure on its input streams. Output tuples though, are then marked as \textit{tentative}, in order
to signal the incorrectness of these results. Once the system has heal from the failure, it tries to reconcile the
state of operators which processed tentative tuples, by re-running their computation with the stable data tuples. This
also means that every source or operator has to buffer all their outgoing tuples, at least for a certain amount of time,
to replay them in case of failure. The goal, in Borealis, is to achieve \textit{eventual consistency}, or the guarantee
that eventually all clients will receive the complete correct results. 

%murty&welsh
Another approach to the management of replicas has been proposed in~\cite{dependable-is-sensing}. The authors take a
different stand point to failure, not considering it as an infrequent event, but as a constant condition within the
system. This is due to the increased size of the infrastructure, which is considered to be deployed at Internet-scale. In this
conditions maintaining consistency among replica becomes even harder. In fact, the authors claim it is better to drop
this constraint all together and to employ instead \textit{free-running} operators. Operator, thus, are allowed to
independently update their internal state as incoming tuples arrive. In case of recovery from failure, for instance, an
operator restarts from scratch, without any knowledge on its previous state. This, on the other hand, means that an
operator's state can diverge from its replicas, due to missing tuples. In general, though, the state of an operator is
based only on the tuples in its \textit{causality window}. Thus, assuming a window size of $w$ seconds and no additional failure, 
the system will regain consistency without the need for explicit synchronization. However, whenever the replicas are out
of sync and produce different results, the system needs to choose which set of results represents the "best" answer.
It can do so in different ways, for instance by choosing tuples generated by the replica with the biggest uptime, with a
majority vote among replicas, or by relying on some \textit{quality metrics} describing the quality of the data. This
process takes the name of \textit{best-guess reconciliation}.

\begin{comment}
\subsection{Query Optimization}

\begin{structure}
	\item Why do we reuse streams? - true scalability
	\item How to reuse streams? - cosmos approach
\end{structure}

COSMOS

\end{comment}

