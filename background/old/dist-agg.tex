\section{Distributed Aggregation}
\label{sec:dist-agg}

The last main category of system which I am going to describe, is concerned with \emph{distributed aggregation}.
These systems are focused on the calculation of aggregate quantities in large systems. In a sense, it is possible to
consider some of these systems as similar to stream proceesing ones, with the difference that these are only interested
in the calculation of aggregation queries~\cite{astrolabe, sdims, dec-netmon}. In this scope, Aggreagtion is intended as a
\emph{summarising mechanism}.  

\subsection{Deterministic approaches}
\subsection*{SDIMS}
\label{sec:sdims}

I will first present the Scalable Distributed Information Management System (SDIMS)~\cite{sdims}. This is a system
concerned with the computation of aggregate quantities about large-scale distributed systems. 

\subsection*{Astrolabe}
\label{sec:astrolabe}

\begin{structure}
	\item System Description
	\item ...
	\item Considerations
\end{structure}

Astrolabe: \cite{astrolabe}

\begin{comment}

\paragraph{Overview}
\label{sec:astrolabe-overview}
\todo{new}

\paragraph{Considerations}
\label{sec:astrolabe-considerations}
\todo{new}

\end{comment}


\subsection{Probabilistic approaches}
Of particular interest for my research is the notion of \emph{imperfect aggregation}. In this case, the system does not
attempt to calculate the exact value of a particular attribute. Instead, it makes an \emph{estimate} which, in some
cases, is also paired with a some confidence metrics. Computing an approximate result instead of the
actual one is much simpler, especially when considering a large-scale distributed system. With massive distribution comes 
massive instability, and calculating a precise value, in some case, might not even be possible. Failure, in fact, has to
be considered not as an isolated event, but as a constant in the system. I will consider two approaches: \emph{gossip-based 
algorithms}, that exploits gossip-style protocols for aggregation, and \emph{network imprecision}, that proposes the
introduction of imprecision metrics to bound the error.

\subsubsection*{Gossip Aggregates}

\emph{Gossip-based} (or \emph{epidemic}) protocols can be used to calculate imperfect aggregate functions in large scale
distributed systems~\cite{gossip-aggregates, gossip-aggregates-large}. In these systems, every node periodically contact 
one or a few nodes and exchanges information with them. Usually the noes chosen for this information exchange are
chosen at random~\cite{cyclon}, but for certain application they get selected among a group of similar nodes, according
to a given metric~\cite{vicinity}. The dynamics of information spread resemble the way information spread within a
group, hence the use of the word \emph{gossip}, or can be seen as similar to the spread of an epidemic in a population. 
The use of these protocols lead to high fault-tolerance and ``self-stabilization'', thus making them particularly useful
in the scope of large-scale distributed systems. 

Gossip-based protocols produce results which are \emph{probabilistic} in nature, but are able to produce results even
when the system is subject to instability and can gracefully scale to a large number of participants. Also, they do not
require error recovery mechanisms, as the system is able to self-organise and to regain balance. For these reasons they
are particularly suitable for peer-to-peer system, where there is no hierarchy and nodes join and leave frequently.
While not producing ``perfect results'', they are able to obtain good estimates without the overhead of other
deterministic approaches, such as the construction of aggregation trees. The total cost of computing these aggregates
is low and be approximated as $O(n\mathrm{log}n)$ messages and $O(\mathrm{log}n)$
rounds~\cite{gossip-aggregates-opt}. The precision of the estimate can be arbitrarily low, depending on the number of
rounds used~\cite{gossip-aggregates-large}.

By employing gossip-based protocols it is possible to calculate a large number of aggregate values in a system. 
The most common examples are \emph{sums} and \emph{averages}, but can also be used to calculate
quantiles~\cite{gossip-aggregates}. Also, it has been shown that they can be employed to calculate wide range of
functions such as counting, generalised means, minimums and maximums, products and also
rankings~\cite{gossip-aggregates-large}.
These protocols can be employed, for instance, to the calculation of the total
number of nodes in the system, or the average load. It is possible to calculate aggregates of any numerical
value held by a node, usually representing a fairly stable property (calculation based on dynamic properties is
possible, but make little sense as an aggregate). For all this reasons, gossip-based aggregates appear to be the best
option to calculate aggregate functions in a large-scale distributed system, and I plan to employ this methods in my
future research.

%As an example to illustrate how a gossip-based aggregate calculation works, I will explain how to determine the total number of nodes in the system. 
%Gossip-Based Computation of Aggregate Information \cite{gossip-aggregates}
%Gossip-Based Aggregation in Large Dynamic Networks \cite{gossip-aggregates-large}
\subsubsection*{Network Imprecision}

Another approach to imperfect aggregation takes the name of \emph{network imprecision (NI)}~\cite{network-imprecision}.
This is a set consistency metric that characterise the quality of the processing in the computation of an aggregate value.
It allows the application to reason if the returned data is to be trusted, allowing, for instance, the filtering of
inconsistent results. NI should be considered then as a ``stability flag'' for the system. Every query result is
enhanced with a set of attribute describing the current state of the system.


Since large-scale distributed systems are seldom completely stable, instead of all-or-nothing stability flag, NI provides three 
metrics: $N_{all}$, $N_{reachable}$ and $N_{dup}$, to provide more flexibility in handling the system instability.
\begin{itemize}
	\item $N_{all}$ is an estimate of the total number of nodes currently in the system.
	\item $N_{reachable}$ is an estimate of the number of nodes whose recent updates might not be reflected in the
current answer.
	\item $N_{dup}$ is an estimate of the number of nodes whose input might have been double counted due to overlay
reconfiguration.
\end{itemize}
Based on these metrics it is possible to infer the stability of the system and thus the quality of the answer. If
$N_{all} = N_{reachable}$ and $N_{dup} = 0$, then the system is completely stable and the query results can be trusted. 
If $N_{reachable}$ is close to $N_{all}$ and $N_{dup}$ is low, the results are calculated on the wide majority of the
inputs and can probably considered valid. If, instead, the $N_{reachable}$ is much smaller than $N_{all}$ and $N_{dup}$
is high, the system is unstable and results should not be trusted.

It is interesting to note that the metrics introduces by NI are themselves aggregate sums. As such they themselves
should be provided with metrics about their quality. This is obviously not the correct way of approaching the problem.
Instead the authors propose to maintain some invariants during the calculation of NI metrics. In particular query
results must be generated by at \emph{at least} $N_{reachable}$ input sources, and \emph{at most} $N_{dup}$ duplicate
inputs due to topology reconfiguration. If the implementation guarantees the maintenance of these two invariants, then
the NI metrics, even though possibly imperfect, can considered as \emph{bounds} on the system imprecision.

The introduction of metrics establishing the quality of the query results allow the user (or the application) to
quantify the achieved quality of processing and can be used by the system to improve its performance. In a similar way we
proposed a quality-centric data model for distributed stream processing systems~\cite{dissp-qdb2009}, a more specific
model that tries also to quantify the information loss occurred in the creation of results. A detailed description of the
quality-centric data model is presented in Chapter~\ref{ch:model}. 

%Probabilistic Counting Algorithms for Data Base Applications \cite{flajolet-martin}

%StatStream: statistical monitoring of thousands of data streams in real time \cite{statstream} | \emph{where does itfit?} 



%\subsection*{Declarative Network Monitoring}
\label{dec-netmon}

\begin{structure}
	\item System Description
	\item ...
	\item Considerations
\end{structure}

Declarative Network Monitoring with an Underprovisioned Query Processor \cite{dec-net-mon}


%\newpage
