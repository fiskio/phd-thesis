\subsection*{HiFi}
\label{sec:hifi}

HiFi~\cite{hifi} is a system designed to manage large supply chains, monitoring the availability and location of goods
making use of RFID tags In the scope a national or international distribution network, keeping track of information
throughout the whole supply chain is a difficult task. Nevertheless, it is widely recognized that an accurate and
real-time monitoring of it can lead to great savings for a company. In this regard the envisioned solution is based on
RFID technology. Every product going in and out a warehouse is monitored and tracked. Because of its nature, RFID
technology is prone to errors. A tag might be scanned twice or not at all, and this poses even more burden on the
monitoring infrastructure. HiFi employs a number of techniques to minimise this error, like data cleaning, detection of
faulty receptors, conversion, calibration and so on. 

The architecture of HiFi is referred to as \emph{high fan}, in which a high number of sources generates data which is
hierarchically processed through successive aggregations, until it reaches the final user. Because of its shape this is
also called a ``bowtie'' system. The reasons for its peculiar configuration are due to two main reasons. First, a
flatter architecture would not be able to cope with the sheer amount of raw data produced by the sources.  Second, this
kind of structure is well suited for the tasks the system was designed for, which is mainly supply chain monitoring. In
this scenario, a company is likely to want to concentrate the final data processing and storage in a central private
facility, but it still needs to push some of the data cleaning, filtering and aggregation close to the input entry
points.  HiFi is built around TelegraphCQ\cite{telegraphcq} and TinyDB\cite{tinydb} It represent a custom solution to a
specific problem, even though its architecture could be suitable also for other purposes.

\paragraph{Considerations}
HiFi can be considered a centralised stream processing system, design for a specific application domanin.
It is built for large supply chain management, monitoring goods moving among warehouses with the use of RFID tags.
It suuports data cleaning, meaning the removal of spurious tag readings, in order to produce more accurate results.
In general, when working with sensor data, having a stage of data cleaning is a good idea. This idea could be
incorporated within the DISSP prototype, with a sanitising component located at entry points of the system.
