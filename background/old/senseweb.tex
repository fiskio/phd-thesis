\subsection*{SenseWeb}
\label{sec:senseweb}

SenseWeb~\cite{senseweb} is an example of ISSP system developed at Microsoft Research. This is a peer-produced sensor
environment, in which many sensors networks are linked together and can be accessed through a uniform interface.  It provides an
intuitive user interface called SenseMap~\cite{sensemap}. This interface displays sensors on a map, where users can
click to select the ones they are interested in.  

A central database is in charge of collecting all the data produced by the sensors and process all the user queries. This
central entity takes the name of \emph{coordinator}.
SenseWeb aims at supporting a wide variety of sensing devices. Their characteristics can be very diverse and so their
communication capabilities.  In order to be able to process data coming from so many different sources, data needs to be
cleaned and transformed into a meaningful format.  This process of input standardisation is carried out by the
\emph{sensor gateways}.  They also enforce stream ownership, giving the sensor contributor the possibility to specify
the sharing policies over his data.  A special kind of gateway is the \emph{mobile proxy}. This has the role of dealing
with a highly volatile swarm of mobile devices, by making the mobility of devices transparent to the applications. Once
the user has specified a location-based constraint, it returns data from all the devices available in the requested
space-time window.  
Another element of the system is the \emph{data transformer}. These are in charge of post-processing
the data, for example by extracting the people count from a video stream. Transformers are indexed at the coordinator
and can be discovered and used by the applications. They work similarly to the gateways, as they convert the data
semantic, but they work of already processed data.

\paragraph{Considerations}SenseWeb represents a first attempt to build a usable stream processing system.
The major limitation of this system has to do with the hourglass configuration in which many
sources and many applications have to deal with a single coordinator.  This can lead to scalability problems when
dealing with a large number of sources and queries. In DISSP we address this problem by employing
a distributed solution for data processing. SenseMap, the resource discovery interface for input sources, is a valuable
tool with a clean and intuitive user interface. If the number of sources become very large, though, it might be
cumbersome for the user to select the relevant ones just by clicking. This is particularly true when the characteristics
of the sensors of interest are known, but not their location. It would be interesting to extend this tool to support
selection of input sources based on their characteristics.

