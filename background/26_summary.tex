\section{Summary}

This chapter presented the relevant background to understand the work presented in this thesis.
It started with a description of a range of stream processing applications, in particular those that have
to deal with large amount of data and can tolerate a certain degree of approximation in their results.
Environmental monitoring is crucial for the timely prediction of possibly disruptive weather phenomena
and for the long term understanding of world-wide climate change. Social media analysis deals with the
processing of user-generated content in online social networks, which is a valuable tool for researchers
to investigate the spread of information through social cascades.
We also presented the most important query models used to express queries in stream processing systems:
the continuous query language (CQL), which extends the relational model for the processing of unbounded
streams, and the boxes-and-arrows model, which employs a visual paradigm to compose queries by creating a
network of operators.
After that, we introduced a system model for stream processing, presenting examples of systems that
pioneered centralised as well as the distributed processing. When dealing with a distributed environment,
the correct allocation of resources across processing sites is crucial. In this regard, we described
efficient resource allocation techniques for both data centre and wide-area deployments.
A major focus of this work is on techniques for the management of overload, in particular
\mbox{load-shedding}.
An overview of the most relevant strategies to discard input data efficiently was provided emphasising
the different issues related to \mbox{load-shedding}.
Finally, there was a description of techniques used to avoid and recover from failures, with a discussion
on the possible consistency and replication models.\\
The next chapter will describe the data model developed in the scope of this thesis which includes a
quality metric that can be used to augment streams and allow the system to detect and measure the effect
of overload. 
