\section{Source Time Window}
\label{sec:eval_stw}

The purpose of this set of experiments is to evaluate if the Source Time Window (STW), introduced in
Section~\ref{sec:stw} is a good approximation of the theoretical concept of Source Information Tuple
Set, introduced in Section~\ref{sec:assumptions}. In a real system it is impossible to know exactly what
tuples will contribute to the creation of a result, thus there is the need for an approximation. 
The idea is to compute \sic values over a large enough time-window that exceeds of at least an order of
magnitude the end-to-end latency of the queries. If the chose STW is too short in fact, in a normally
loaded system, it would result in a \sic value heavily fluctuating around 1. 

We evaluate the STW approximation as we increase the processing and network end-to-end latency.
To this end, we gradually grow the number of query fragments from one
to six---thus increasing the processing delay---and we deploy each fragment in a separate Emulab 
node---thus increasing the network latency---while we keep the time-window
fixed to 10~seconds. Overall, we perform six
experiments (\ie one for each number of query fragments) and for each we deploy
10 queries from each type of the complex workload mix. In all cases, \sys performs
perfect processing and the average result SCR is shown in Table~\ref{table:time-window}.
Results show a very low fluctuation of values, well within the expected range, thus meaning that the STW
approximation implemented in the \sys system correctly captures perfect processing in all cases.
\begin{table}[h]
  \centering
  \begin{tabular}{c|c||c|c}
    \hline
        fragments & SCR$\pm$std & fragments & SCR$\pm$std \\ \hline
	1 & 1.032$\pm$0.008 & 2 & 1.036$\pm$0.011 \\ \hline
	3 & 1.020$\pm$0.066 & 4 & 1.018$\pm$0.069 \\ \hline
	5 & 1.036$\pm$0.034 & 6 & 1.009$\pm$0.063 \\ \hline
  \end{tabular}
  \caption{Time-window approximation for various fragments.}
  \label{table:time-window}
\end{table}
