\section{Summary}

This chapter presented the experiments performed to evaluate the advantages of employing the \sic quality
metric in a stream processing system. All experiments were carried out using the \sys prototype
system described in Chapter~\ref{ch:system_design}.  
The first set of experiments looked at the correlation between \sic values and the correctness of
results.
Even though the \sic quality metric was not designed as a direct measure of the accuracy of results,
our experiments showed that, for many classes of queries, there is a good correlation, confirming that a
high value of the \sic metric indicates a high confidence in the correctness of results.
The second set of experiments focused on the evaluation of the fair shedding policy, designed to
exploit the \sic metadata to allocate resources among queries evenly and compares its performance with
that of a random shedder. Our results showed that employing the fair shedding policy leads to a more even
distribution of \sic values for the delivered output tuples.
Finally, a set of experiments was devised to investigate the correlation between resource cost and
the processing quality of the results. 
They showed that it is possible to strike a balance between the achieved \sic values
of the results and a reduced resource cost for the processing infrastructure.
