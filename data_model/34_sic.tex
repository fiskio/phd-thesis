\section{Source Coverage Ratio}
\label{sec:sic}

%n the previous section I showed how the previously stated assumptions about the quality-centric data
%model can be applied to some generic queries. The goal was to provide an intuitive description about how
%the \sic values are calculated for different query types. In this section I will provide a formal
%definition about \sic calculation.

This section provides the definition of a new quality metric called \emph{Source Coverage Ratio
(\sic)}. A complete set of equations is derived for
its calculation, following the reasoning expressed in the assumptions presented in
Section~\ref{sec:assumptions}. The section finishes with a discussion of the benefits of the \sic metric
when employed by a stream processing system.

\subsection*{Introduction}

Typically a stream processing system does not have global knowledge if all the
available data was processed correctly. 
Therefore, we propose that the impact of failure should be monitored and accounted for. 
In our model, we augment streams with a new metric called \textit{Source Coverage Ratio}
(\sic), which quantifies the amount of information in a tuple, thus recording its
importance for the current query results. The \sic metric is added to streams in the form of metadata,
whose value is calculated by the query operators and is inversely proportional to the level of failure.

A tuple should convey information about the amount of failure experienced during its creation. There
should be a way to indicate if the data carried by a tuple includes all the available
information, making it 100\% accurate, or if some information was lost in the process, thus reducing
its quality. \textit{Source Coverage Ratio} tries to achieve
this goal by measuring the amount of lost information.

The system can use the \sic metric to evaluate the importance of a tuple towards the final query result. 
When the system is overloaded, it has to decide which tuples should
be discarded in order to recover. In this case, \sic values can be used to assess the
individual value of tuples and guide the system to a selection that helps improve the quality of the
results.

% In a single query scenario the system is able to reason about the amount of information carried by
% tuples. By dropping the tuples with the lowest values, it minimises the impact of tuple loss on the query
% results.
% Even in the absence of failure, it is possible that some tuples aggregate more information than others and should
% be preserved with higher priority. In case of failure, the system should discard low valued tuples
% instead.
When only a single query is deployed, the system can minimise the impact of overload by selecting the
tuples to be discarded among those with the lowest \sic values.
When dealing with a system that allows the concurrent execution of multiple queries, \sic values
help discard tuples in a way that affects all queries in the system equally.
We refer to this as \textit{fair shedding}. 
%and it will be further described in Section~\ref{sec:fair-shedding}.

\subsection*{Data Model Formalisation} 
\label{sec:sits}
The purpose of the \sic metric is to capture the amount of information that contributed to the creation
of an output tuple. In the absence of failure, the \emph{perfect value} of the \sic metric is equal to
the number of sources. A reduction from this value is caused by failure during processing. 

Since it is not possible to know in advance which tuples will contribute to the creation of an output
batch of tuples, the final \sic value is not calculated based on individual tuples but instead over a
window of tuples. The amount of information is defined over a time interval so that the final result \sic value
is the sum of the individual \sic values of all tuples produced within the corresponding time window.
Every tuple within the interval is assigned the same value, as for Assumption 5 (Tuple~Equality). All
sources assign the same total \sic value in a time window, as for~Assumption~4~(Source~Equality).

For example, a source producing 100 tuples in a given time interval
assigns a \sic value of 1/100 to each tuple. Another source producing only 20 tuples during the same
time interval assigns individual \sic values of 1/20. The result tuples obtained after processing
would have an aggregated \sic value of 2 without failure.
We refer to the complete set of tuples produced by the sources in a given time interval as the
Source Information Tuple Set.

% \begin{definition} [Source Information Tuple Set]
% %\label{sec:sits}
% The source information tuple set $\mathcal{T}^{S}$ of a result tuple $t_{R}$ is the set of source tuples
% defined by a function $f^{-1}: \mathbb{T}^{R} \rightarrow \mathbb{T}^{S}$ when applied over $t_{R}$, \ie
% $f^{-1}(t_{R})=\mathcal{T}^{S}$.  $\mathbb{T}^{S} = \{ \mathcal{T}^{S}_{s}~|~s \in \mathcal{S}\}$ where
% $\mathcal{T}^{S}_{s}$ denotes the set of source tuples in $\mathcal{T}^{S}$ generated from source $s$.
% \end{definition}
\begin{definition} [Source Information Tuple Set]
%\label{sec:sits}
The source information tuple set $\mathcal{T}^{S}$ of a result tuple $t_{R}$ is the set of source tuples
that contributed to the creation of $t_{R}$ (\ie the set of tuples that were processed by the query
function $f_Q : \mathcal{T}^{S} \rightarrow t_{R}$ to generate $t_{R}$).
\end{definition}
%%\mnote{the definition is taken from the Themis paper but seems a bit too complicated to me.}
In our query model, every result tuple $t_{R}$ is associated with the set of source tuples
$\mathcal{T}^{S}$ that contributed to its creation. We consider a query as a black-box with a \emph{query
function} $f_Q$ that maps source tuples to result tuples. The \textit{source information tuple set} of a
result tuple is the set of all source tuples. 

This concept is central to the definition of the \textit{Source Coverage Ratio} (\sic) quality
centric metric. The intuition behind it is that a result tuple is considered to be perfect when no
information from the sources was lost, meaning that all tuples in its source information tuple set (and
any derived tuples) were processed correctly. If one of the tuples is lost, either due to failure or
deliberate load-shedding, the information contained in the result tuple is not perfect, and its \sic
value is decreased accordingly.

\begin{definition}[Source Coverage Ratio] {
The source information content (\sic) of all result tuples for query $Q$, 
denoted as $\sic_Q$, measures the contribution of all source tuples, belonging to the source
information tuple set $\mathcal{T}^{S}$, in the result tuple set $\mathcal{T}^R$. It is calculated as: 
\begin{align} 
		\sic_Q = \sum_{t_R \in \mathcal{T}^R} t_{R}^{SCR} = \sum_{t_{src} \in \mathcal{T}^{S}}
		t_{src}^{SCR},
\end{align}
where $t_{src}$ are source tuples in $\mathcal{T}^{S}$ used for the calculation of the
result tuples $t_{R}$. In particular, $t_{src}^{SCR}$ shows the contribution of a source tuple from 
source ${src\in\xspace S}$ to the perfect result and is defined by:
\begin{align}
		t^{SCR}_{src} = \frac{1}{|\mathcal{T}_{src}^{\mathcal{S}}||\mathbb{S}|}
\end{align} 
where $|\mathcal{T}_{src}^{\mathcal{S}}|$ is the total number of tuples in the Source Information Tuple
Set of source $src$, and $|\mathbb{S}|$ is the total number of sources in the query. }
\end{definition}
\vspace{-5pt}
Equation~(3.1) states that the total \sic value of all result tuples for a given query, is equal to the
sum of the \sic values of all tuples in the Source Information Tuple Set. 
% It aggregates all the \sic values of all source tuples that contributed to its creation. 
Since a query can have multiple terminal operators, each potentially producing more than one result
tuple at the time, we consider the \emph{sum} of all the result tuples produced by all terminal
operators.
If all tuples are correctly processed, the
resulting \sic value is 1. A lower value indicates that some information loss happened
during the query execution.

Equation~(3.2) describes how source tuples are assigned their individual \sic values. First, a value of 1
is assigned to the totality of tuples produced by a source in its Source Tuple Information Set. This
means that all sources are considered to contribute equally to the final result, regardless of the amount
of tuples that they produce during an interval.
Since every tuple in this set is considered to contain the same amount of information, this value is
divided by the number of tuples produced by the source during a time interval.
Thus, the \sic value of an individual source tuple $t_{src}$ is inversely proportional to the number of
tuples in $|\mathcal{T}_{src}^{S}|$. Furthermore, it is necessary to normalise by the number of sources
$|\mathbb{S}|$ connected to a query, dividing the initial \sic values by the number of sources.
In this way, all resulting \sic values are in the interval $[0,1]$, which makes it possible to compare
the quality of different queries.
\vspace{-10pt}
\subsubsection*{\sic Propagation from Source to Result Tuples}
\vspace{-5pt}
Up to now, we considered queries as black-boxes and discussed only the relation of \sic values between
the source and result tuples. In order for the formal description to be complete, it is important to
understand how intermediate \sic values are calculated at individual operators and how these propagate
within the system.

Source tuples from $\mathcal{T}^{S}$ are processed by the query operators in the set $\mathcal{O}$, which
produce new derived tuples. These tuples flow to the next downstream operator until they reach a terminal
operator that outputs result tuples.
More formally, for a given intermediate derived tuple $t_{out}^{o}$ and for an operator $o \in
\mathcal{O}$, we define $\mathcal{T}_{in}^{o}$ to be the set of all input tuples to an operator $o$ required for the
generation of $t_{out}^o$.
Similarly, $\mathcal{T}_{out}^{o}$ denotes the set of derived tuples produced by operator $o$ after the
processing of $\mathcal{T}_{in}^{o}$.

The \sic value of a derived tuple $t$, processed by operator $o$, (\ie $t \in \mathcal{T}_{out}^{o}$) is:
% \begin{align} t_{SCR} = \frac{\displaystyle\sum_{t \in
% \mathcal{T}_{in}^{o}}{t_{SCR}}}{|\mathcal{T}_{out}^{o}|}.
% \end{align}
\begin{align} 
	t^{SCR}_{out} = \frac{1}{|\mathcal{T}_{out}^{o}|} \cdot \displaystyle\sum_{t_{in} \in
	\mathcal{T}_{in}^{o}}{t^{SCR}_{in}}
\end{align}
This means that the set of derived tuples produced by an operator contains all the information from the
input tuples that the operator processed. This information is considered to be distributed equally over
all the output tuples: the total value is divided by the number of output tuples.

The way that the \sic values are calculated is recursive: it is the same if a
single operator or a complete query, seen as a complex black-box of operators, are considered. The sum of
the information for the input is always propagated to the output tuples and equally divided among them. 
\vspace{-10pt}
\subsubsection*{Relation of \sic to Query Processing Quality} 
The \sic values of result tuples lie in the $[0,1]$ interval: \\
(a) A value equal to $1$ indicates that the complete set of $\mathcal{T}^{S}$ tuples was
used during processing and that the result is perfect. \\
(b) A value of $0$ indicates that all source tuples from $\mathcal{T}^{S}$
were discarded or lost during processing. \\  
(c) A value between $(0,1)$ indicates a degraded result, meaning that only a subset of the tuples
contained in the Source Tuple Information Set was used for computation. A small value means a
large information loss, while a value close to 1 indicates an almost perfect result. 
\subsubsection*{Model Benefits}

\textbf{Query Performance.} Even though \sic values provide a good estimate of the quality of the
processing, it should be noted that their value is not linked directly to the absolute error of the query
results. In general it is not possible to compute error bars on the delivered results based on this
metric.
\sic values are an indication to the user about the amount of failure that occurred during the generation
of a query result. It is then up to the user to decide if the delivered answer should be considered valid
or should be discarded. 

Nevertheless \sic values are a good indication about the quality of the results. The
knowledge of the query semantics may allow a user to correlate \sic values with an absolute result error. 
The simpler the query semantic, the easier it is to establish such a correlation. 

\textbf{Completeness of Results.} Using \sic values, it is possible to compare the results of the same
query at different times to evaluate the quality of the results in terms of information
completeness. A result tuple $t^R_1$ delivered a time $t_1$, compared to another result tuple $t^R_2$
delivered at time $t_2$, contains more information if and only if $t^{SCR}_1 > t^{SCR}_2$.

A user can monitor the status of a computation by observing the changes in \sic values of a
query at different times. A decrease in \sic values means that more information was
discarded, and there may be a need to act upon it. Reasons for a sudden reduction may be:\\
(1) the failure of one or more processing nodes, requiring the migration of some operators and
recovery of the failed infrastructure; or\\
(2) a sharp increase in input rates, requiring the increase of processing resources to maintain \sic
values within acceptable bounds.

\textbf{Fair Resource Allocation.} Using \sic values, it is possible to compare the performance
achieved by different queries running concurrently on a shared processing platform. A fair system would
try to equalise the quality of processing of running queries, allocating resources in such a way as to
provide results with similar \sic values for all queries.

When the system experiences \emph{overload} and needs to discard a certain number of tuples, it can use
the \sic values to determine which tuples to drop and which tuples to preserve. This process is achieved
through \emph{fair shedding} and will be further explored in Section~\ref{sec:fair-shedding}.
\vspace{-10pt}
