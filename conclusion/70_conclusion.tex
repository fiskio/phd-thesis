\chapter{Conclusion} \label{ch:conclusion}
\vspace{-15pt}
In a world in which the production of data continues to grow, data stream processing systems (DSPSs) play
an important role by allowing the computation of continuous queries over streams of data in real-time.
Their applications include, among others, the processing of environmental sensor data, allowing
scientists to monitor a wide range of natural phenomena, and the analysis of user-generated content
shared through social media, offering new ways of interpreting human interactions.

The large amount of available data makes the correct provisioning of DSPSs difficult. This is
due to the unpredictable volatility of input data rates and the costs associated with the purchase or
rental of computing platforms. A large amount of input data requires processing resources that may
not be available to a user. To acquire resources, it is possible
to use a federated resource pool, in which different parties contribute their resources to
create a larger shared processing infrastructure, or to rent the required processing resources on demand
in a cloud environment. In both cases, however, it may not be feasible to obtain the necessary resources,
either due to practical issues or economical constraints.

As a consequence, overload may be a common condition in DSPS systems.
Traditionally, overload is considered a rare, transient condition but our perspective is that it is a
normal operating condition. Instead of trying to achieve perfect processing at all times, a DSPS under
overload should account for its impact and report any processing degradation to users. For many
applications, an imperfect result may still be valuable to the user.

Under overload, the lack of processing resources forces the system to discard a certain fraction of the
input data, a process called load-shedding. The choice of which tuples to discard is critical to the
correct functioning of the system.
Instead of discarding tuples at random, we propose an intelligent load-shedding
policy which selects tuples in a way so that all queries are penalised equally.

\subsection*{Quality-Aware Stream Processing Under Overload.}
%\vspace{-15pt}
This thesis presented a novel quality-centric data model for stream processing. It augments
data streams with a quality metric called Source Coverage Ratio (\sic). 
The definition of this metric follows the reasoning outlined in a set of
assumptions and considerations, making it applicable to a DSPS with
generic operators.
The \sic metric tracks the failure during processing, providing an indication of the
quality of the computed results. 
It allows the user to reason about the result quality and the system to
implement intelligent policies for overload management.

A prototype stream processing system called \sys was developed to evaluate the benefits offered by the
\sic metric. It was designed to process data efficiently under overload and provide continuous feedback
to the user about the achieved quality of processing of queries.
It exploits the \sic metric to degrade gracefully the quality of results for all queries. A particular
focus is the efficiency of \mbox{load-shedding}, following a design that allows for the
flexible implementation of semantic shedding policies. 
The algorithm that decides which tuples to keep and which to discard leverages the information contained
in the \sic values of tuples to implement a fair load-shedding policy.

We defined fairness in a DSPS based on the \sic values achieved by queries (\ie a system is fair if all
queries achieve an equal quality of processing) as expressed through the \sic value.
Based on this definition, the \sic metric was used to implement a fair load-shedding policy, which
detects and handles an excess of processing load by discarding a portion of the input tuples. This allows
all queries to achieve an equal quality of processing, reducing the differences in terms of the \sic
values of their results.
System resources are allocated fairly among queries, regardless of their size or characteristics.
This is particularly useful in a shared processing infrastructure, such as a federated resource pool, so
that all users receive a fair share of the processing resources.

We performed a set of experiments to explore the characteristics of the \sic metric and its
applicability in the context of semantic load-shedding. The experimental results indicate that there is a
good correlation between the error that occurs in the results and their associated \sic values. The
fair shedding policy was compared with a random rejection of input data. Results show that the
intelligent policy outperforms the random one, both in terms of the average \sic values as well as the
dispersion of the values. The calculation of the \sic metric showed good scalability properties,
maintaining a stable performance when varying the amount of processing resources and the number of
deployed queries. Finally, we showed that it was possible to expose a trade-off between the reduction in
cost per query and the reduction in \sic value when a user wants to reduce the processing costs and is willing to accept approximate results.
\\
\subsection*{Future work} 
%The possibilities offered by the \sic metric are not completely explored. 
Directions deserving further investigation include the application of the quality-centric data model to
the dynamic allocation of the optimal amount of processing resources in a cloud deployment, the use
of free-running replicas to increase the ability of the system to withstand failure and the augmenting of
operators with custom error functions to better correlate \sic values and the accuracy of results.

\paragraph{Dynamic Resource Allocation.}
In a cloud deployment, the quality-aware data model could be used to scale the amount of
rented resources at run-time, detecting a temporary shortage of resources in case of a sudden increase of
the input rates, and also signaling an excess of processing capacity causing unnecessary costs.
Based on the \sic value of results, it is either possible to \emph{scale-out}, increasing the amount of
rented processing nodes when the \sic values of results do not reach a given threshold, or to
\emph{scale-in}, reducing the amount of processing nodes in order to reduce the rental cost, when the
amount of processing resources is excessive.

Scaling-out resources allows to find the optimal amount of processing nodes to rent in order to achieve a
given quality of processing. When a query is first deployed, an horizontal operator decomposition starts
by allocating a modest amount of processing resources, decomposing the main operator into a small number
of copies. It then checks if the number of allocated processing nodes is sufficient to achieve result
\sic values that are above a user specified threshold. If the result \sic values are below this
threshold, more processing nodes are allocated, increasing the number of parallel subqueries of the decomposition.
This process is repeated until the achieved result \sic values go above the minimum threshold. In this
way, it is possible to correctly provision the amount of resources needed to run a certain query,
thus minimising the processing rental cost.

The \sic metric could also be employed to scale-in resources, deallocating processing nodes when they
are not needed. When the achieved result \sic values are well above the acceptable minimum
value, the system can estimate if there is an advantage in reducing the number of processing nodes
allocated for the current computation. For example, if the amount of input data is reduced, the current allocation of
resources may become excessive and too costly.
The system could start migrating queries from a node to the rest of the infrastructure and
finally release it, reducing the resource costs. It would repeat this process as long as the result \sic
values remain above a given threshold.

\paragraph{Replica Management.}
The proposed quality-centric data model can be also used to increase the system ability to withstand
failure with the use of \emph{free-running replicas}~\cite{dependable-is-sensing}. This thesis was mainly
concerned with issues arising in a system failing due to overload. Other kinds of failure that may occur
in a DSPS include node crashes and network partitions. To mitigate the impact of failure caused by
overload or other means, a DSPS should replicate the most important operators of a query. By running
multiple copies of the same operators in parallel, more copies of the same result are computed at
different sites. In case of failure of a node, the system can use the result calculated at another
replica. In case of overload, the system can choose the result with the highest processing quality.

Under failure, it is typical for different replicas to produce different results. The internal state
of operators may not be consistent due to the missing inputs. A standard approach is to employ a
\emph{eventual consistency} model, in which the internal state of the failed replica is restored after
failure by replaying the missing tuples. However, this may not be feasible, due to
the lack of resources.\\
We propose instead to adopt a more relaxed consistency model, in which operators continue processing
without any attempt of reconcile the internal state of a failed replica.
If this relaxed consistency model is employed, the DSPS system must choose among the different outputs
produced by the replicas, deciding which one to deliver to the user. In this situation, the \sic
metric offers a valuable indication about the quality of the output computed by the replicas. By
capturing the amount of failure that occurred during the creation of tuples, it allows the system to
select the correct output to deliver by choosing the one with the highest quality of processing (\ie
that has the higher \sic value) and thus is closer to what would have been produced in absence of
failure.
\vspace{-10pt}
\paragraph{Custom Error Functions.}
A design choice that was made for the quality-centric data model is to ignore the processing semantics of
the individual operators. Being operator agnostic allows the model to be applicable to any type of
queries, without being restricted to a limited set of operators. Even though a link between the quality of
processing and the accuracy of results was shown in the evaluation of the model, there is no direct
correlation between the result \sic values and the error caused by overload.

An extension to the model would be to allow users to submit custom error functions for operators. In this
way, the model could calculate the exact error for the computed results. For a range of generic
operators the error function could be already provided by the system, leaving the option of modifying it
to the user. For custom operators instead, it would be necessary to submit it together with their
implementation. 


