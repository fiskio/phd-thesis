\addcontentsline{toc}{chapter}{Abstract}

\begin{abstract}
\vspace{-15pt}
Data stream processing systems (DSPSs) compute real-time queries over continuously changing streams of
data.  A stream is a potentially infinite sequence of tuples, or timestamped data items.  
% In contrast to traditional databases, in which queries are issued over stored data, results in DSPSs are
% generated continuously as new data enters the system. 
% \vspace{6pt}\\
% % 
The constant increase in data volume renders the provisioning of a DSPS challenging, requiring computing
resources that may not be available or too costly to purchase. Even if a user opts for a cloud deployment
model, thus renting all resources on demand, acquiring sufficient resources may still not be possible due
to the financial cost.

In the future, we can expect the development of processing infrastructures, in which different
parties cooperate to create federated resource pools. 
These kinds of deployments, in which many parties pool together their resources, are subject to a
phenomenon similar to the \emph{tragedy of the commons}. Since every party tends to
consume more resources than they contribute, the amount of available resources is always scarce.
%
\vspace{6pt}\\
%
% Overload can be considered a type of failure because the system is not able to fully carry out the
% required computation. A system operating under constant overload is thus subject to continuous failure.
For these reasons, \emph{overload} should be considered a common operating condition for such DSPS and
not an exception.
In such situations, the system needs to discard some of its input data, which is an operation called
\emph{load-shedding}. 
When an overloaded system performs \mbox{load-shedding}, the choice of how much and what to discard is
crucial for the correct functioning of the system.
\vspace{6pt}\\
%In such a resource constraint scenario, failure due to overload is not a transient condition.
%
Many streaming applications are able to produce useful results even after some data has been discarded	
during the processing.
Examples of such applications are meso-scale weather prediction, better tornadoes and hurricanes forecasting
and real-time social media monitoring.
%Meaningful results can still be produced, even when some data is lost during the computation. 
An approximate result may still be useful to the user, as long as it is delivered with a low latency and
contains some information about its quality.
%In many cases, an imperfect result is better than no result at all. 
%  In this scenario of constant approximation, the system should continuously report to the user
% the estimated impact of \mbox{load-shedding} on the quality of the results.
\vspace{6pt}\\
% Instead the system should be designed taking this into account, implementing approximation strategies 
% that allow the system to still produce meaningful results while providing the user with a measure of 
% the achieved quality of service. 
We propose a new model for federated stream processing under overload. The system constantly
estimates the impact of overload on the computation and reports to the user the achieved processing quality. We introduce a
 quality metric called \textit{Source Coverage Ratio (\sic)}.
%to augment data streams, and a set of
% formulas to calculate its propagation within the system. This has been designed to work for any kind of operator and for both the \textit{fan-in} and the \textit{fan-out} families of queries.  
This can be used by the user as an indicator for the achieved processing quality and by the system to
implement intelligent shedding policies and to better allocate the system resources among users.
% When an overloaded system performs \textit{\mbox{load-shedding}}, the choice of how much and what to discard is
% crucial for the correct functioning of the system. 
The \sic quality metric allows the implementation of a \textit{fair shedding} policy, giving an equal
processing quality to all users without penalising individual queries. 
\vspace{6pt}\\
%
% We develop these ideas as part of a research prototype called DISSP, the Dependable Internet-Scale
% Stream Processing engine. With it we explore the issues related to overload
% management and fair resource allocation.
%, with the aim of building a system able to operate under constant failure.
We experimentally show that augmenting streams with the \sic metric allows the system to make
better \mbox{load-shedding} decisions, leading to more accurate results for many types of queries.
It also allows the user to reason about the amount of processing resources that are needed to run a
given query, striking a balance between the quality of the delivered results and the resource cost.
%We show that is still possible to achieve a good approximation of the results
%while sensibly reducing the costs for the user.

\end{abstract}
