\subsection*{Compiled Operators and Tuples}
\label{sec:templates}
In our system tuples and operators are implemented as \emph{compiled POJO objects}. This choice was made
taking performance into consideration, since other options, based for instance on \emph{reflection}, were considered too
slow and cause of a potential performance bottleneck. To instantiate a custom tuple or operator object,
based on the query requirements, a text \textit{template} is completed with the information contained in the XML file
describing the query producing a new .java file, which is then compiled to bytecode by a run-time compiler. 

\paragraph{XML Query Representation} Queries are submitted to the system using and XML representation. An
XML query file contains the complete description of the query. It specifies what kind of tuples will be
processed, providing a description of the \emph{tuple schemas} so that operators are aware of the
number, type and name of the fields contained in the tuples they process. It also contains the list of operators
implementing the query semantic. Every operator is represented by an XML block, containing its
description. Every operator needs to be specialised before it can be used within a query. A generic
implementation is provided in a \emph{template file}, which acts as a blueprint for the operator, which
is then finalised with the data included in the XML block, allowing the correct instantiation of the
operator.
The information needed to complete an operator templates includes its \emph{name, parameters} and an indication about its \emph{following operator}. Every operator either declares itself as a
\emph{terminal} operator, by declaring no downstream operators, or as an \emph{intermediate} operator by
declaring that its output should be fed in input to another operator. This allows the system to
reconstruct the complete query graph. 
The information contained in the XML representation is used to generate the complete .java files
describing theon customised tuples and operator that will be used in the query. This can then be compiled
and instantiated so that the query can begin processing. 

\paragraph{Templates} are semi-complete \textrm{.java} files, representing the skeleton of a class. 
They are used for the efficient instantiation of customised \textit{tuples} and \textit{operators}.
All tuples share some common characteristics, but are differentiated by the number of fields, their names
and types. The same is true for all operators belonging to the same type. For instance Average operators
are all equals when it comes to the processing semantic, but they differ about the name and type of the
field to be averaged. So a generic blueprint for these object comes from the generalisation of a specific
instance, where specific names and types are replaced by some place text tokens called \emph{place
holders}. Using the information contained in their XML description it is possible to substitute these
place holders with the correct details, transforming a template into a complete \textrm{.java} ready for
compilation.
Place holders are all capital keywords preceded by a dollar sign, in the
form ``\texttt{\$PLACEHOLDER}''. They are meant to be completed using the information provided in the XML
file describing the query and the compiled into actual POJO objects with the desired characteristics. For
this transformation the system employs a \textit{CharSequenceCompiler}, which takes in input a string
containing the content of a completed \textrm{.template} file and produces an instance of the customised
object.
This allows the system to operate always on compiled bytecode, granting the maximum performance of
execution together with the flexibility of working with customised versions of tuples and operators
tailored to the specific query requirements.
\lstset{
  basicstyle=\ttfamily,
  columns=fixed,
  showstringspaces=false,
  commentstyle=\color{white}\upshape
}

\lstdefinelanguage{XML}
{
  morestring=[b]",
  %morestring=[s]{>}{<},
  morecomment=[s]{<?}{?>},
  stringstyle=\color{BrickRed},
  identifierstyle=\color{NavyBlue},
  keywordstyle=\color{ForestGreen}, 
  morekeywords={type,name}% list your attributes here
}
\noindent\begin{minipage}{\textwidth}
\begin{lstlisting}[language=XML,label=lst:opxml,caption=XML description of an Average operator]
	<operator name="MyAvgCpu" type="Average">
	    <next name="MyOutput"/>
	    <parameter name="tuple"    value="simpleSchemaONE" />
	    <parameter name="field"    value="cpu"/>
	    <parameter name="groupby"  value="idx"/>
	<operator>
\end{lstlisting}
\end{minipage}

\exblock{Listing~\ref{lst:opxml} shows the XML description of an \emph{Average} operator. In the first
line the operator is characterized as being an instance of class ``Average'', described in the correspondant
.template file, and it is given the name of ``MyAvgCpu'. Then there is the declaration of a \emph{next}
operator, this means that this is not a terminal operator and thus its output should be delivered to a
single local operator named ``MyOutput''. Next are 3 parameter definitions, in the form $\langle name,
value \rangle$. The system will look for the ``\$NAME'' placeholder and will replace it with the string
given by ``value''. Once the substitution has taken place, the .template file becomes a complete .java
source file and is then compiled by the \emph{CharSequenceCompiler}.}
	
	
%--------------------------------- 
\lstset{
  basicstyle=\ttfamily,
  columns=fixed,
  showstringspaces=false,
  commentstyle=\color{white}\upshape
}

\lstdefinelanguage{XML}
{
  morestring=[b]",
  %morestring=[s]{>}{<},
  morecomment=[s]{<?}{?>},
  stringstyle=\color{BrickRed},
  identifierstyle=\color{NavyBlue},
  keywordstyle=\color{ForestGreen}, 
  morekeywords={type,name}% list your attributes here
}
\begin{lstlisting}[language=XML,label=lst:tuplexml,caption=XML description of a Tuple]
	<schema name="simpleSchemaONE">
	    <field type="long"   name="ts"  />
	    <field type="double" name="idx" />
	    <field type="double" name="tmp" />  
	</schema>
\end{lstlisting}


As described in Section~\ref{sec:templates}, Tuples are implemented using a template file. One parent
class called ``Tuple'' is provided, containing the basic logic common to all tuples, and every other
tuple class derive from this. In the XML query description, the user specifies a schema and a name for
the tuples that will be used by the query and the system creates a new tuple class, with the name and
fields required. A generic tuple \texttt{.template} file is filled, substituting some placeholders with
the provided data, producing the \texttt{.java} source file of the new tuple class.

Listing~\ref{lst:tuplexml} shows the XML description of a Tuple object with 3 fields: one \texttt{long}
for the timestamp named \textit{ts}, and two \texttt{double}, one with a numerical identifier \textit{idx}
and the other carrying a temperature reading \textit{tmp}. These values will be inserted in the tuple
.template in correspondance with the ``\texttt{\$FIELD}'' placeholder, transforming it into a complete
.java source file. The compiled object will contain 3 public fields with the type and name specified in
 the XML listing.
 
%--------------------------------- 
 
 Listing~\ref{lst:opxml} shows the XML description of an \textit{Average} operator. In the first line the
operator is characterized as being an instance of class ``Average'', described in the correspondent
.template file, and it is given the name of ``MyAvgCpu'. Then there is the declaration of a \textit{next}
operator, this means that this is not a terminal operator and thus its output should be delivered to a
single local operator named ``MyOutput''. Next are 3 parameter definitions, in the form $\langle name,
value \rangle$. The system will look for the ``\$NAME'' placeholder and will replace it with the string
given by ``value''. Once the substitution has taken place, the .template file becomes a complete .java
source file and is then compiled by the \textit{CharSequenceCompiler}.
		 
\lstset{
  basicstyle=\ttfamily,
  columns=fixed,olding,
  showstringspaces=false,
  commentstyle=\color{white}\upshape
}

\lstdefinelanguage{XML}
{
  morestring=[b]",
  %morestring=[s]{>}{<},
  morecomment=[s]{<?}{?>},
  stringstyle=\color{BrickRed},
  identifierstyle=\color{NavyBlue},
  keywordstyle=\color{ForestGreen}, 
  morekeywords={type,name}% list your attributes here
}
\noindent\begin{minipage}{\textwidth}
\begin{lstlisting}[language=XML,label=lst:opxml,caption=XML description of an Average operator]
	<operator name="MyAvgCpu" type="Average">
	    <next name="MyOutput"/>
	    <parameter name="tuple"    value="simpleSchemaONE" />
	    <parameter name="field"    value="cpu"/>
	    <parameter name="groupby"  value="idx"/>
	<operator>
\end{lstlisting}
\end{minipage}

In the next sections there will be an overview of the main classes of operators provided by the DISSP
prototype. \textit{Window Operators} provide transformations over window sizes, holding the input of an
operator until a certain condition is reached. \textit{I/O Operators} are the gateways to the system, in
input they provide the conversion from external to system tuple representation, in output they deliver
the query results to the user. \textit{Network Operators} allow the inter-node communication, so that a
query can be partitioned and distributed onto several nodes. Finally there will be a list of the main
\textit{Data Manipulation Operators}, these are the ones actually performing the processing on the data,
and are equivalent to the \textit{relation-to-relation} operators in CQL.

\mnote{Where should I place the operator list? Here seems a bit dispersive}

\subsubsection*{Window Operators}
\label{sec:window-op}
Window Operators provide a way of regrouping tuples, changing the size of their container batches and are
equivalent to the \emph{stream-to-relation} operators in CQL. 
In CQL a stream is a continuous entity and needs to be broken down into a series of relations through a
\emph{stream-to-relation} operator before operators can process it. In our system a stream is a purely 
abstract entity, since batches already are series of tuple snapshots, but window operators are still used
to group the input of an operators, so that every new input batch contains tuples with a precise
semantic. The regrouping of tuples can be done according to their \emph{timestamp}, according to the
values of a \emph{field}, or according to their \emph{position} in the stream.

\paragraph{Sliding Windows}


\paragraph{Time Window}
A \emph{Time Window} defines a temporal interval and produces an output batch containing all the tuples
received during the interval. For instance a time window of interval ``1 minute'' outputs a batch of
tuples every minute, containing all tuples received in the previous 60 seconds. The incoming tuples
accumulate in the internal buffers until the current time interval is over, then they are grouped into a
new batch that is sent in output and a new time interval begins. 

\paragraph{Field Window}

\paragraph{Count Window}
A \emph{Count Window} defines a new output batch composed by an exact number of tuples. For instance a
count window of size N always produces batches of N tuples in output. If an input batch contains M
tuples, with M greater than N, it gets broken down into a series of batches of size N, until the
remaining tuples are less than N. These remainder tuples sit in the internal operator buffer until they
are merged with a new incoming batch. When a new batch arrives it is merged with the tuples still present
into the internal buffer from the previous execution and, if their total amount is greater than N, a new
output batch is produced.   

			
From the standpoint of a user, implementing a new custom operator boils down to the definition of
a new java class, extending either the \textit{BlockingOperator} or the \textit{NonBlockingOperator} class,
which implements the \texttt{execute()} method. To make this class generic it is then necessary to
convert it into a \textit{template} file. This is done by replacing some fields with a generic placeholder
which will then be specified in the XML description of this operator.

	
	
		\subsubsection*{I/O Operators}
		\label{sec:input-output}
In DISSP there are two main kinds of InputDevices operators. The first connects to an external source and 
then receives tuples as they become available. The other kind instead generate a \emph{synthetic
workload}, employing a file containing the log of a previous stream of tuples, which are replayed at a
certain rate. This second option is particularly useful when performing system tests and in experiments
where different strategies has to be evaluated, for instance comparing the performance of two load
shedding strategies. A Source can start multiple InputDevice operators at once, so that the same
component can act as a gateway to multiple real sources.
A more detailed description about the different types of \emph{input devices} implemented within the
DISSP system is provided in Section~\ref{sec:input-output}.		
			\paragraph{Source Input Device}
			\paragraph{Output Device}
		
		\subsubsection*{Network Operators}
		\label{sec:network-op}	
			\paragraph{Remote Sender}
			\paragraph{Remote Receiver}
		
		\subsubsection*{Data Manipulation Operators}
		\label{sec:data-op}				
			\paragraph{Filter}
			\paragraph{Average}
			\paragraph{Covariance}
			\paragraph{Join}
			\paragraph{TopK}
			\paragraph{Union}
			\paragraph{Min}
			\paragraph{Max}