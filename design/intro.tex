%\mnote{new title}
%\mnote{4.0 provides a general introduction to chapter}
This chapter introduces the \emph{DISSP stream processing system}, a prototype developed to implement and
evaluate the quality-centric data model described in the previous chapter. 
It describes some of the novel features of its design and their implementation.
It shows how the system employs the Source Information Content (\sic) metric to \emph{perform efficiently
under overload} and to \emph{provide feedback} to the user about the achieved quality of processing of queries.

The system has been designed from the ground up to be distributed, allowing the processing of queries to
span onto many computing nodes.
It was particularly targeted at environments with a \emph{constrained amount of resources} that can not
easily be scaled. Examples of such deployment settings include, for example, \emph{federated resource
pools} with different authorities administering different processing sites; or \emph{cloud deployments}
where the resources are rented and it may be more cost effective to voluntarily reduce the amount of
processing resources, trading an acceptable reduction in the correctness of results for a substantial 
reduction of the operational costs. 

\emph{Overload} is assumed to be the normal condition of operation of the system and not a transient,
rare event. 
%The system assumes \emph{overload} to be normal condition of operation and not a transient condition. 
The \sic values of tuples are used as a hint to their importance, so that the system can perform an 
\emph{intelligent selection} of the tuples to shed. This allows the creation of \emph{flexible shedding
policies}, as it will be illustrated in the next chapter.
This chapter describes the implementation choices made to provide a reliable and efficient
processing of queries in such an hostile environment.
It also presents and overview of the different system-wide components and their interaction to deploy and
run queries. 

\pagebreak
