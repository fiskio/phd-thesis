\chapter{Introduction}
\label{ch:introduction}

%what is a DSPS
Data stream processing systems (DSPSs) compute real-time queries over
continuously changing streams of data~\cite{8-reqs}. A stream is a potentially infinite sequence of tuples, or timestamped
entities~\cite{cql}. Differently from traditional databases, in which queries are issued over stored
data, in DSPSs queries are issued first and results are generated as new data enters the system.
This allows the output of real-time updated results based on changing data
streams. Real-time in this context refers to the ability of delivering results \emph{as they become
available}. Stream queries are therefore \emph{continuous}~\cite{cont-queries}.

% ISSPs
Internet-scale stream processing (ISSP) systems represent the next step in the evolution of stream
processing~\cite{dissp-challanges}.
Similarly to how search engines make static web data useful to users, an ISSP system reliably
collects, filters and processes stream data from potentially thousands of data sources on behalf of many
users. Such systems should support a large number of queries, possibly running for extended periods of
time. Data should be processed in a decentralised fashion, distributing the computation over several data
centers.

In this context, the notion of \emph{sensor} is not only limited to a device measuring some physical
quantity but also refers to a more extended range of devices. \emph{Physical sensors} are devices taking
measurements of some physical quantity at interval time, \eg a temperature sensor deployed in a
sensor network. On the other hand, \emph{software sensors} take measurements of non-physical events,
such as social media events.

Figure~\ref{fig:dissp} shows the deployment of an ISSP system. Sources, either physical or
software sensors, are located at different locations around the world (\ie
illustrated as weather towers). Data streams are depicted as arrows. Several data centres, also located
at different locations, are used to process these streams in a decentralised fashion. Many users can concurrently issue queries, which the
ISSP system tries to satisfy using the available processing resources.
\clearpage
\begin{figure}[t!] 
	\centering 
	\includegraphics[width=0.8\textwidth]{img/tesi/mydissp2} 
	\caption{Graphical representation of an Internet-scale stream processing system.}
	\label{fig:dissp}
\end{figure}
We can expect the rise of globally distributed ISSP systems in the future, in which several parties
contribute processing resources to be shared by all users. Such federated resource pools join clusters
distributed across the globe, each administered by a local authority. Users deploy queries processing data from sources located all over the world in real-time. MaxStream~\cite{maxstream} is a
federated stream processing infrastructure designed to support real-time business intelligence
applications, and Cosm~\cite{cosm} is a system aiming to construct a world-wide network of
intelligent devices for the \mbox{Internet-of-Things}.
The amount of available input data renders the provisioning of processing resources
challenging because it may be too costly to purchase or rent the necessary infrastructure. 
A world-wide federation of resources allows the creation of large infrastructures, in which
 users take advantage of each others processing capabilities when needed.
%in time of need, while allowing the use of theirs when unused. 

% -------------------------------------------------------------------------------------------------------

\section{Applications}

A federated stream processing system can support the concurrent execution of a
large number of queries, submitted by multiple users. 
% The amount of available input data renders the provisioning of the processing resources
% difficult as it may be too costly to purchase or rent the necessary infrastructure.
The variety of queries that a stream processing system must support is broad and can be the basis for a
large number of applications.
Among these, two interesting application domains, which will be used as examples, deal with queries
analysing \emph{sensor data} and \emph{social media} streams.


\paragraph{Sensor data.} 
Advances in cheap micro-sensing technology led to an increase in the availability and resolution of
sensor data.
In the future, the number of sensing devices will reach a point at which everything of material
significance is ``sensor tagged'', reporting its state and position in real-time.
One area that has recently flourished in this field is environmental monitoring. 
Understanding how and why the climate is changing is a scientific priority and will be crucial
to plan future policies.
Sensor networks can also be employed to monitor weather phenomena that are potentially
catastrophic, such as hurricanes and tornadoes. 
The creation of large-scale weather sensing infrastructures allows to identify
these phenomena before they become disruptive, thus enabling an early response to mitigate their impact.

\paragraph{Social media.} 
Social media represent a large source of user generated data, which offers new interesting	
possibilities for data mining. 
Data streams are populated with status updates, location check-ins, photos, videos, etc. This results
in an ever increasing influx of everyday experiences that users decide to share. 
The amount of available data is vast and opens the doors to a new class of streaming
applications.
Dealing with the totality of the generated data may be too costly, and the analysis of samples, capturing
a certain percentage of the original stream, may be sufficient.
Discarding a fraction of the input data allows a substantial reduction of the processing
resources needed to execute queries, thus reducing operational costs.
What is important is that this reduction in quality of the generated results is quantified and reported
to the user. 
This leads to a trade-off between the processing cost and the quality of processing achieved by
the system.

In this application domain, a result may
be acceptable even if imperfect, as long as there is an indication of the degree of quality
degradation. The judgement about the quality of the delivered results should be left to the end user,
while the system should provide continuous feedback about the achieved quality of processing.
% -------------------------------------------------------------------------------------------------------

\section{Problem Statement}

In certain domains, the over-provisioning of a stream processing system is not possible, such as
in the case of a federated resource pool, or it may be too costly, such as in a cloud deployment.
When the amount of resources needed for perfect stream processing is not available, 
it may be possible to deliver meaningful results by processing only a subset of
the input data.
In many cases, an imperfect result is better than no result at all. 


The constant increase in data availability allows the computation of a richer variety of stream
processing queries but also renders the provisioning of the system more difficult and costly.
A large amount of input data requires processing resources that may not be available to the user or
too costly to purchase.
% The cost of purchasing and administering the needed processing infrastructure can be too high and outside
% the financial possibilities of a user. 
In order to obtain the necessary processing capacity without having to own the entire infrastructure, it
is possible to opt either for a \emph{federated resource pool} or for a \emph{cloud deployment}.

In a federated resource pool, many parties contribute their resources to create	 a larger shared
infrastructure, which can be used by all according to their needs. In such a scenario, each institution
is only responsible for the costs of purchasing and administering their own processing resources, while
being able to use the global set of resources. 
% An example of such a shared infrastructure is
% PlanetLab, where many academic institution joined a consortium and created a geographically distributed
% general purpose processing infrastructure that spans the entire globe. 
These kinds of deployments, in which many parties pool together their resources, are subject to a
phenomenon similar to the \emph{tragedy of the commons}~\cite{Hardin68}. If every party tends to
consume more resources than what they contributed, the amount of available resources becomes scarce.
%TODO
%Another problem with such deployments is that, when resources are scarce, it is only possible to
%increase the provisioning of the local domain, while the other processing sites are under the control of
% third parties.

Another way to acquire a large amount of processing resources is to rent them on demand~\cite{cloud}. 
In a cloud deployment model, the required processing nodes are rented from a cloud
provider for the time needed. In this way, the cost of purchasing and maintaining the infrastructure
is outsourced, and the user pays only for the incurred resource usage. 
This allows the acquisition of a large processing capacity for a reasonable cost.
The amount of resources needed, however, can be large and the rental cost may become too expensive. 
Therefore, even if a user opts for a cloud deployment model, perfect
processing may not be feasible due to these financial constraints.

The loss of information that determines the quality reduction of results is due to either the loss of
tuples or their intentional discarding.
Processing nodes can fail and some may become temporarily unavailable due to a network failure,
thus causing a loss of a portion of the processed tuples. 
In these cases, the stream processing system should track the loss of information and
quantify its impact on the quality of the results. Therefore, the user should be provided with a quality
metric that reports the achieved quality of processing so that they can assess the ``goodness'' of
the provided output.

A common reason that leads to the reduction of processing quality is the deliberated
shedding of a fraction of tuples. This happens when the rate at which tuples are received by a
processing node is greater than the rate at which it is able to process them. The throughput of the node
saturates and cannot increase any more. The excess tuples need
to be discarded, which is referred to as \emph{\mbox{load-shedding}}~\cite{load-shedding}.
In order to overcome an overload condition, it is possible to increase the amount of processing
resources but, as explained above, this may not be feasible. Instead, it may be acceptable to sustain
the overload condition if an acceptable quality of the results can be maintained.

In a resource constrained deployment with a large number of users and queries, an efficient allocation of
resources becomes critical. The problem is how to allocate resources fairly so that all queries achieve
a similar quality of processing, regardless of their nature. The system should be able to reason
about the relative performance of queries so that, when it has to perform \mbox{load-shedding}, it
discards tuples in a fair fashion. Having a metric that captures the achieved quality of processing for
all queries allows the system to implement a shedding policy that penalises queries fairly.
In this way, the quality of the delivered results is equalised, and queries receive an optimal amount of
processing resources.% according to their needs. 

This thesis attempts to address a set of research questions.
\enumerate{
\item Is it possible to track the loss of information and report it to the user?
\item In a DSPS is there a correlation between the quality of the processing and
the correctness of the delivered results? 
\item When the DSPS has to choose which tuples to discard in an overload
condition, is it possible to design a fair resource allocation so that all queries experience the same quality of processing? 
\item In a cloud deployment, in which resources are rented according to demand,
what is the trade-off between the cost of processing and the quality of the delivered results? } 
% -------------------------------------------------------------------------------------------------------

\section{Quality-Aware Stream Processing}

A federated stream processing system that operates under constant overload has to be designed with the
ability to evaluate its achieved quality of processing. When
the amount of input data exceeds the processing capabilities of a node, the system has to discard a given amount of
input tuples. By discarding new tuples for a certain amount of time, the load on the
system can be reduced and the overload condition overcome. Such an approach, however, does not take into
account the nature of the discarded tuples. 

When overload is not an exceptional condition but is considered normal,
the system needs to be able to reason about the information carried by the individual tuples of each query. 
This allows the system to implement a \emph{semantic} shedding policy, in which the tuples to be
discarded are chosen according to a given set of rules.
By augmenting tuples with a metadata value expressing their importance to the query computation, it is
possible to select tuples based on their information content and avoid random data loss. 
This thesis proposes a new quality metric to quantify the amount of information captured by a data tuple,
thus tracking the degradation of processing quality due to overload.

% To address these challenges, this work proposes two main contributions: the introduction of a
% quality-metric called \emph{Source Coverage Ratio (\sic)} and its use for the implementation of a
% \emph{semantic shedding policy} with a fair allocation of resources.s crucial
The choice of what tuples should be discarded and what should be kept affects the quality of the
results. A federated stream processing system has to decide which tuples are more important for the
processing and prioritise them over other tuples of lower quality. Not all tuples contain the same
amount of information and thus have a different value to the query. For example, a tuple containing a
value aggregated over a large number of sensors is likely to be more valuable than a tuple containing a single sensor reading. The system should
be able to reason about the information content of tuples in order to identify the most valuable tuples to
be spared from shedding. To do so, we introduce a \emph{quality metric} that captures the
quantity of information that went into the creation of a tuple and thus its value. 
Using this metric, the system is able to implement a \emph{semantic shedding policy}, which is based on
the quality of tuples and retains the most valuable tuples for the computation.

% -------------------------------------------------------------------------------------------------------
\vspace{-10pt}
\section{Contributions}
\vspace{-5pt}
\paragraph{SCR quality metadata metric.}
In order to allow the system to reason about the achieved quality of processing, this thesis
proposes a quality metric called \emph{Source Coverage Ratio (\sic)}. 
It provides an estimate of the amount of information contained in a tuple and
thus its importance for the query results. 
It is added to streams in the form of metadata and its value is calculated by the query operators.
The system uses the \sic metric to implement intelligent resource allocation policies.
Chapter~\ref{ch:data_model} introduces the quality-centric data model for federated stream processing
system based on the \sic metric. It presents the assumptions behind the \sic metric and explains
how it is calculated.
\vspace{-3pt}
\paragraph{Semantic quality-aware load shedding.}
When an overload condition is considered normal, it becomes important to choose intelligently which input
tuples to discard during \mbox{load-shedding}. 
If the rate at which tuples are received is higher than the rate at which they can be
processed, a node has to discard a certain fraction of its input
tuples. 
Augmenting streams with the \sic quality metric provides an indicator of the amount of
information captured during the creation of a tuple, thus quantifying its importance to query processing.
When performing \mbox{load-shedding}, the system can leverage this knowledge to realise a semantic
load-shedding policy (\ie an algorithm that chooses the set of tuples to discard according to
specific criteria). 
Chapter~\ref{ch:load_shedding} presents the design and implementation of a fair load-shedding policy,
which tries to allocate system resources proportionally to the requirements of queries so that the
achieved quality of processing is similar for all queries.
\vspace{-3pt}
\paragraph{Prototype system design.} 
The concepts introduced by the quality-aware data model have been realised in the \sys stream processing
prototype system. 
This system is designed to make use of the \sic quality metric to
operate under overload while tracking the quality of processing achieved by each query. 
The system implements a semantic load-shedding policy that provides a fair amount of resources to all
queries.
The prototype is described in Chapter~\ref{ch:system_design}, which shows how the basic concepts of the
data model are realised. It presents the system-wide components interacting at a high level as well as
the low level architecture of each processing node.
\vspace{-3pt}
\paragraph{Experimental evaluation.}
We report results on the correlation between the degradation of \sic values and the
correctness of query results. Even though the \sic quality metric is not designed as an accuracy metric,
for many aggregate queries, there is a good correlation between \sic values and correctness.
Having introduced semantic load-shedding policies, we investigate the advantages of employing the \sic
quality metric for \mbox{load-shedding}. In particular, we compare a random shedder and the fair shedding
algorithm introduced in Chapter~\ref{ch:load_shedding}. We find that the fair shedding policy
leads to better resource allocation among queries for many classes of queries.
% Chapter~\ref{ch:data_model}, when presenting the quality-centric data model and the \sic metric,
% introduces the concept of \emph{Source Information Tuple Set}. This is an abstract entity that was
% approximated with a \emph{Source Time Window} when implementing the prototype. A set of experiments was
% designed and validated the hypothesis that the source time window construct represents a good
% approximation of the source information tuple set.

% A set of experiments was also designed to test if some of the theoretical constructs that have been
% introduced by the data model can be implemented using an approximation and if this is good enough to be
% employed.
Finally, we explore the correlation between \sic degradation and cost. In
this scenario, the user voluntarily accepts a reduction of result quality in order to reduce the amount
of required resources and thus the cost of a cloud deployment. Our results show that there is a trade-off between a reduced \sic value of results and the costs
of renting cloud resources.

% -------------------------------------------------------------------------------------------------------
\section{Thesis Outline}

The rest of this thesis is organised as follows. Chapter~\ref{ch:background} presents the background
material for this research work. It describes some applications of stream processing, focusing on
those that can tolerate imperfect processing. It introduces different streaming data models and the
basic concepts of stream processing. It presents different approaches for overcoming overload, focusing
on \mbox{load-shedding} strategies.

Chapter~\ref{ch:data_model} introduces our quality-centric data model, in which streams are augmented
with the \sic metadata metric. It states the assumptions used to derive the quality metric and presents
an overview of the different categories of queries. The chapter ends with the formalisation of the model
and some examples of its application.

Chapter~\ref{ch:system_design} presents the design of the \sys stream processing system, a research
prototype that implements the proposed quality-centric data model. It explains how the abstract concepts
from the model can be implemented as a part of a real-world system. We describe the design of the
prototype and show how the \sic metric is calculated and used.

Chapter~\ref{ch:load_shedding} focuses on overload management by describing our approach for semantic
fair \mbox{load-shedding}. It introduces the concept of fair resource allocation among
queries and of a semantic load-shedding policy that implements it. It then describes the algorithm used
to discard tuples on overloaded nodes so that all queries achieve a similar \sic value for their results.

Chapter~\ref{ch:evaluation} presents a set of experiments designed to evaluate properties of the
\sic quality metric. It investigates its correlation with the accuracy of results and compares a random
shedding policy with our proposed fair shedding algorithm. Finally, it looks at the trade-off between
cost and \sic value degradation.

Chapter~\ref{ch:conclusion} concludes the thesis and includes a discussion of future research.
