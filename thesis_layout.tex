%\documentclass[10pt,a4paper]{article}
%\usepackage[latin1]{inputenc}
%\usepackage{amsmath}
%\usepackage{amsfonts}
%\usepackage{amssymb}
%\usepackage{makeidx}
%\usepackage{fullpage}
%\begin{document}

% PETER AWAY NEXT WEEK

%\title{Layout}

\section{Introduction}
\begin{enumerate}
	\item Introduction to stream processing and failure
	\item Roadmap of the thesis
\end{enumerate}
\emph{1 week. In various iterations, I think it's going to take a few days.}

\section{Background}
\begin{enumerate}
	\item List of many other similar system
	\item Highlights of similarities and differences
\end{enumerate}	
\emph{I will base this chapter on the previous work done for the second draft transfer report, updated and improved.\\ NEEDS FLESHING OUT}
\emph{4 weeks in total. I see this as low priority work, where every week I take 1 day just to focus on one paper/system. I think in this way it will be more fun than just go for a month only in reading mode.}

\section{Quality-centric data model}

\subsection{Problem Analysis}
\begin{enumerate}
	\item Streaming relational model
	\item Necessity of a data model to capture failure
\end{enumerate}	

\subsection{Sensor data queries}
\begin{enumerate}
	\item Queries as graph with multiple sources and single result (A)
	\item Running example with sensor data
\end{enumerate}

\subsection{Fan-out queries}
\begin{enumerate}
	\item Queries as graphs with multiple endings, multiple aggregates from single source (B)
	\item Running example with Twitter data
\end{enumerate}

\subsection{Data Model}	
\begin{enumerate}
	\item Derive a unified data model based on the 2 above scenarios
	\item Examples of useful application of the datamodel
\end{enumerate}

\emph{Here I will present the 2 query classes and derive the data model.\\NEED TO DEFINE UNIFIED MODEL}
\emph{3 weeks. This chapter needs a lot of work since we still have to come up with a revised model that can unify the two scenarios. Before coming to London I am going to think about this so that we can together about it in the first meeting. Hopefully this part will be defined by the end of my staying.}

\section{Design chapter}
\begin{enumerate}
	\item Introduction to stream processing and failure
	\item Introducing our concepts of stream, tuple, query, etc..
	\item failure and overload can be addressed in the same way using the data model
	\item Implementing a stream processing engine deigned to cope with missing information
	\item ...
\end{enumerate}

% ~5 sections per chapter, ~3 section 

% scalability? distribution? coordinator -> single node, distributed? etc...

% what scenario? federated, cluster, cloud?

\emph{Main implementation chapter, describing our way to stream processing and with many implementation details.}
\emph{4 weeks. Once a piece of the system is finalised I can write about it, about 2 days every time. } 
\section{Load Shedding}
%algorithm + approach implementing it
\begin{enumerate}
	\item Need to discard some input
	\item Random shed Vs Intelligent shedding
	\item The design of an efficient and intelligent shedder
	\item Evaluation of different strategies in different scenarios
	\item ...
\end{enumerate}
\emph{This chapter will be largely based on the work done for the current paper}
\emph{4 weeks, continuous. For this chapter I think it's better to work on it until is over. The time here will be spent mostly measuring performance and writing about it, since I already have a good understanding of the shedding process.}

\section{Evalutation}
% evaluation goals
% fairness, scalability, failure tolerance, etc...
\begin{enumerate}
	\item Evaluation with query type A
	\item Evaluation with query type B
\end{enumerate}
\emph{I will evaluate the data model in both cases.}
\emph{8 weeks. This is the part that I forecast taking the longer. Deciding the evaluation experiments, implementing them and writing about the results is a very long process in my experience.}
\section{Conclusion}
\emph{1 week. For this chapter as for introduction let's say a week, which is also good as buffer.}
%\end{document}
